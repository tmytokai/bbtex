\documentclass[a4paper]{article}
\usepackage{bbtex}
\usepackage{amsmath}

% 用紙サイズは自分で好きなように決める
\pagestyle{empty}
\hoffset = -1in
\voffset = -1in
\topmargin = 15mm
\headsep = 0mm
\oddsidemargin = 15mm
\evensidemargin = 15mm
\textwidth = 180mm
\textheight = 267mm

\begin{document}
%
% 初期化
% 1番目の引数は乱数の種なので日時等を入れると良い
\initbb{2016010901}{生産工学科1年}{前期期末試験}{テスト}{平成28年1月9日}{高専太郎}
%
% コメントを外すと解析モードになる
% カッコの中は受験人数
% \analysismode{10} 
%
% 解答項目分布の解析結果の読み込み
% データは analysis.sh と analysis.ods を組み合わせて作る
% 詳しくはマニュアル及び analysis.sh と analysis.ods 内の説明を参照
% \analysiscomment の中に入れておくと通常時に余計なメモリを食わなくて良い
\analysiscomment{% 実際のデータを使用したい場合は以下の例を消して analysis.ods 内のコードをコピー
\distrib{1}{0}{2}\distrib{2}{0}{3}\distrib{3}{0}{3}\distrib{4}{0}{1}\distrib{5}{0}{0}\distrib{6}{0}{2}
\distrib{1}{1}{7}\distrib{2}{1}{4}\distrib{3}{1}{1}\distrib{4}{1}{0}\distrib{5}{1}{8}\distrib{6}{1}{2}
\distrib{1}{2}{1}\distrib{2}{2}{10}\distrib{3}{2}{4}\distrib{4}{2}{3}\distrib{5}{2}{0}\distrib{6}{2}{3}
\distrib{1}{3}{0}\distrib{2}{3}{2}\distrib{3}{3}{1}\distrib{4}{3}{4}\distrib{5}{3}{1}\distrib{6}{3}{1}
\distrib{1}{4}{0}\distrib{2}{4}{1}\distrib{3}{4}{1}\distrib{4}{4}{2}\distrib{5}{4}{1}\distrib{6}{4}{2}
}{}
%
\bbtitle % タイトル表示
\par\bigskip
%
\analysiscomment{  % 解析モード時に表示
\par\medskip
\begin{center}
\begin{tabular}{|c|c|c|}
\hline
平均点 & 標準偏差 & 受験人数 \\
\hline
50.1 & 1.05 & 10 \\
\hline
\end{tabular}
\par\medskip
\begin{tabular}{|c|c|c|c|c|c|}
\hline
60〜50 点& 49 〜 30 点& 29 〜 0 点 \\
\hline
1 & 7 & 2 \\
\hline
\end{tabular}
\end{center}
}{}
%
\bbsection{以下の選択問題を答えよ。}
%
\begin{makeq}{10}{0}{0} % 2番目の引数が0の時は順番通りに解答項目を表示
日本の首都を選べ。
\incor{ソウル}\cor{東京}\incor{ワシントン}\incor{バンコク}\incor{北京}
\end{makeq}
%
\begin{makeq}{10}{1}{5} % 2番目の引数を1にすると解答項目をランダムにシャッフルして表示
アメリカの都市を2つ選べ。
\cor{ニューヨーク}\cor{シカゴ}
\incor{ソウル}\incor{バンコク}\incor{モスクワ}\incor{パリ}
\end{makeq}
%
\bbsection{以下の穴埋め問題を答えよ。}
%
\begin{center}
アメリカの首都は\clabel[f]{イ}{ワシントン}、中国の首都は\clabel[f]{ロ}{北京}である。% 正答にラベルを付ける
\end{center}
%
\par\bigskip
\begin{makeq}{10}{0}{0}
(イ)に入る都市
\incor{ソウル}\incor{東京}\cref{イ}\incor{バンコク}\incor{北京}
\end{makeq}
%
\begin{makeq}{10}{0}{0}
(ロ)に入る都市
\incor{ソウル}\incor{東京}\incor{ワシントン}\incor{バンコク}\cref{ロ}
\end{makeq}
%
\bbsection{以下の数式を完成させよ。}
%
\begin{align*}
2+2 &= 4 \quad \clabelqnum[f]{0}{\times} \quad 1 \\ % 正答に問題番号のラベルを付ける
&= 8 \quad \clabelqnum[f]{1}{\div} \quad 2
\end{align*}
\par\bigskip
%
\begin{makeq}{10}{1}{5}
(\bbqnum)に入る数式を選べ。
\crefqnum % makeq 環境外 でラベル付けした正答を使用
\incor{$\int$}\incor{$\pm$}\incor{$\sin$}\incor{$\cos$}\incor{$\div$}
\end{makeq}
%
\begin{makeq}{10}{1}{5}
(\bbqnum)に入る数式を選べ。
\crefqnum % makeq 環境外 でラベル付けした正答を使用
\incor{$\int$}\incor{$\pm$}\incor{$\sin$}\incor{$\cos$}\incor{$\times$}
\end{makeq}
%
\totalscore
%
\analysiscomment{}{ % 解析モードで無い時に実行
%
\makeanswersheet{4} % 解答用紙
%
\makeanswersheet[c]{4} % 模範解答
%
\makeanswerfile{bbtex.txt} % ブラックボード用の問題バッチファイル
% タブコードの代わりに文字列 _TAB_ を出力しているので
% sed -e "s/_TAB_/\t/g" bbtex.txt > batch.txt の様に_TAB_をタブに置き換える
%
}

\end{document}
